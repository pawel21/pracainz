\chapter{Wstęp} \label{rozdz.wstep}
Niniejsza praca dotyczy stoworzenia układu pomiarowego do badania charakterystyk laserów półprzewodnikowych w laboratorium fotoniki
 Politechniki Łódzkiej.\\
Celami pracy jest:
\begin{itemize}
\item Stworzenie interfejsu pomiarowego w laboratorium fononiki do badania charakterystyk laserów półprzewodnikowych przy wykorzystaniu oprogramowania
Open Source.
\item Zbadanie charakterystyk czterech laserów półprzewodnikowych.
\end{itemize}
W celu stworzenia interfejsu pomiarowego musimy potrafić sterować sprzętem laboratoryjnym za pomocą komputera.
Do tego celu dobrze nadaje się oprogramowanie
Open Source takie jak język programowania Python oraz system operacyjny Linux.
Szukając informacji o wykorzystaniu języka Python do komunikacji ze sprzętem pomiarowym można zauważyć pewną lukę,
którą moja praca ma cel wypełnić przynajmniej w części.
Na stronie firmy Thorlabs, której sprzęt jest używany w laboratorium fotoniki,
brak jest programów do komunikacji ze sprzętem dla platformy Linux.
Dostępne są jedynie wysokopoziomowe API (interfejs do komunikacji z danym urządzeniem, biblioteką) do systemu Windows
 oraz istnieje możliwość użycia środowiska LabVIEW.
LabVIEW jest graficznym środowiskiem programistycznym używanym do sterowania sprzętem pomiarowym
oraz do zbierania danych pomiarowych. Niestety LabVIEW jest programem płatnym. Ja w swojej pracy będę używać środowiska Linux,
w którym wszystko jest plikiem, także sprzęt połączony przez USB z komputerem, dzięki czemu możemy się z nim komunikować
używając standardu komend SCPI przez wykorzystanie wywołań systemowych, co zostanie opisanr w pracy.

Dzięki mojej pracy możliwe będzie wykonywanie pomiarów charakterystyk laserów półprzewodnikowych w sposób łatwy oraz w
krótkim czasie.
W swojej pracy przedstawiam charakterystyki laserów krawędziowych i laserów VCSEL, które zmierzyłem za pomocą programów napisanych
w trakcie pisanie pracy.
Charakterystyki dają nam ważne informacje o laserze, dzięki nim możliwe jest między innymi wyznaczenie prądu progowego oraz
określenie sprawności danego lasera.

Praca jest podzielona na dwie części: pierwsza składa się z opisu przygotowania eksperymentu, sposobu sterowania urządzeniami laboratoryjnymi
za pomocą programu napisanego w języku Python dla platformy Linux (Ubuntu).
Druga część pracy opisuje badanie laserów półprzewodnikowych na podstawie danych uzyskanych
za pomocą programu przedstawionego w pierwszej części. Do wykreślenia charakterystyk wyjściowych oraz wyznaczenie sprawności
badanych laserów używam skryptów napisanych w języku Python.