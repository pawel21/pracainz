\chapter{Wstęp} \label{rozdz.wstep}
Niniejsza praca dotyczy zakresu inżynierii oprogramowania sprzętu pomiarowego w celu wykorzystania go w badaniu charakterystyk laserów półprzewodnikowych w laboratorium fotoniki Politechniki Łódzkiej. \\

Głównym celem pracy stworzenia interfejsu
pomiarowego w laboratorium fononiki do badania charakterystyk laserów półprzewodnikowych przy wykorzystaniu oprogramowania
open source. \\
W ostatnich latach obserwuje się gwałtowny rozwój wykorzystania oprogramowania
open source w codziennej pracy naukowej. Coraz większą popularność zdobywa język Python.
Od dawana podstawowym systemem operacyjnym używanym przez naukowców są różne odmiany systemu Unix.
Jest to spowodowane dostępnością wielu narzędzi(C, Python,
Gnuplot) których naturalnym środowiskiem jest środowisko Linux, ułatwiającym pracę naukową. Inną
zaletą środowiska Unix jest możliwością korzystanie z linii poleceń, która ułatwia wiele
zadań. Szukając informacji o wykorzystaniu języka Python do komunikacji ze sprzętem pomiarowym można zauważyć pewną lukę,
którą moja praca ma cel wypełnić. Korzystając
z strony oraz dokumentacji firmy Thorlabs, której sprzęt jest używany w laboratorium
fononiki, należy zauważyć brak programu do komunikacji ze sprzętem na platformie Linux.
Dostępne są jedynie wysokopoziomowe API do systemu Windows oraz możliwość użycia
LabVIEW. Minusów środowiska Windows nie sposób wymienić w kilku zdaniach. Program
LabVIEW jest programem płatnym. Rozwiązaniem wszystkich problemów jest użycie środowiska Linux,
gdzie wszystko jest plikiem, także sprzęt połączony przez usb z komputerem, dzięki czemu możemy się z nim komunikować używając standardu komend SCPI przez
wykorzystanie wywołań systemowych. Dzięki temu mamy możliwość dostępu do wszystkich możliwych funkcji sprzętu
pomiarowego bez ponoszenie kosztów. Umożliwia nam to
sterowania sprzętu za pomocą komputera oraz wizualizacje i analizę danych w sposób, jaki
potrzebujemy. A wszystko to dzięki połączeniu możliwości środowiska Linux oraz języka Python \\
Głównymi celem mojej pracy wykorzystanie oprogramowania
open source takiego jak Python systemu Linux do stworzenia stanowiska pomiarowego w celu bdania laserów półprzewodnikowych.
Korzystając z tych technologi mam zamiar stworzyć interfejs pomiarowy na platformę Ubuntu
w laboratorium fotoniki. \\
Dzięki mojej pracy możliwe będzie wykonywanie w szybki sposób charakterystyk laserów półprzewnodnikowcyh.
Charakterystyki te dają nam ważne informacje o laserze, dzięki nim możliwe jest określenie prądu progowego dla laserów krawędziowych,
określenie ich sprawności. Za pomocą mojego stanowiska pomiarowego możliwe także będzie badanie ile uzyskuje się mocy z lasera przy danej mocy aplikowanej. \\
Praca jest podzielona na dwie części: jedna składa się z opisu przygotowania eksperymentu, komunikacji oraz sterowaniem urządzeniami laboratoryjnymi
 za pomocą programu napisanego w języku Python. Druga część pracy opisuje badanie laserów półprzewodnikowych na podstawie danych uzyskanych
 za pomocą programu przedstawionego w pierwszej części programu. Do wykreślenia charakterystyk wyjściowych oraz wyznaczenie sprawności
 badanych laserów używam skryptów napisanych w języku Python.