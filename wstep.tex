\chapter{Wstęp} \label{rozdz.wstep}
Niniejsza praca dotyczy zakresu inżynierii oprogramowania sprzętu pomiarowego w celu wykorzystania go do badania charakterystyk
laserów półprzewodnikowych w laboratorium fotoniki Politechniki Łódzkiej. \\
Celami pracy jest:
\begin{itemize}
\item Stworzenie interfejsu pomiarowego w laboratorium fononiki do badania charakterystyk laserów półprzewodnikowych przy wykorzystaniu oprogramowania
open source.
\item Zbadanie charakterystyk czterech laserów półprzewodnikowych.
\end{itemize}
W celu stworzenia interfejsu pomiarowego musimy potrafić sterować sprzętem laboratoryjnym za pomocą komputera, do tego celu dobrze nadaje się oprogramowanie
Open Source takie jak język programowania Python oraz system operacyjny Linux.
Od dawana podstawowym systemem operacyjnym używanym przez naukowców są różne odmiany systemu Unix.
Jest to spowodowane dostępnością wielu narzędzi C, Python, Gnuplot, których naturalnym środowiskiem jest środowisko Linux.
%Inną zaletą środowiska Unix jest możliwością korzystanie z linii poleceń, która ułatwia wiele zadań.
Szukając informacji o wykorzystaniu języka Python do komunikacji ze sprzętem pomiarowym, można zauważyć pewną lukę,
którą moja praca ma cel wypełnić przynajmniej w części.
Przeglądając stronie firmy Thorlabs, której sprzęt jest używany w laboratorium fotoniki,
brak jest programów do komunikacji ze sprzętem dla platformy Linux.
Dostępne są jedynie wysokopoziomowe API do systemu Windows oraz istnieje możliwość użycia LabVIEW.
LabVIEW jest graficznym środowiskiem programistycznym używanym do sterowania sprzętem pomiarowym
oraz do akwizycji danych pomiarowych. LabVIEW jest programem płatnym. Ja w swoje pracy będę używać środowiska Linux,
gdzie wszystko jest plikiem, także sprzęt połączony przez USB z komputerem, dzięki czemu możemy się z nim komunikować
używając standardu komend SCPI przez wykorzystanie wywołań systemowych. Dzięki temu mamy możliwość dostępu do wszystkich możliwych funkcji sprzętu
pomiarowego w sposób, w jaki potrzebujemy z możliwością pisania skryptów.

Dzięki mojej pracy możliwe będzie wykonywanie pomiarów charakterystyk laserów półprzewodnikowych.
W swojej pracy przedstawiam charakterystyki laserów krawędziowych i laserów VCSEL.
Charakterystyki dają nam ważne informacje o laserze, dzięki nim możliwe jest wyznaczenie prądu progowego oraz
określenie sprawności danego lasera.

Praca jest podzielona na dwie części: pierwsza składa się z opisu przygotowania eksperymentu, sposobu sterowania urządzeniami laboratoryjnymi
za pomocą programu napisanego w języku Python dla platformy Linux(Ubuntu).
 Druga część pracy opisuje badanie laserów półprzewodnikowych na podstawie danych uzyskanych
za pomocą programu przedstawionego w pierwszej części. Do wykreślenia charakterystyk wyjściowych oraz wyznaczenie sprawności
badanych laserów używam skryptów napisanych w języku Python.