\chapter{Podsumowanie}
\section{Rezultat pracy}
W ramach pracy stworzyłem program do sterowania pomiarami charakterystyk elektryczno-optycznych laserów półprzewodnikowych.
Program stworzony jest
w dwóch wersjach: skryptowej oraz okienkowej. Program został napisany w języku Python w sposób obiektowy, co ułatwi
pracę nad nim w przyszłości. Praca moja wypełniła lukę, którą był brak programu do sterowania sprzętem firmy Thorlabs
na platformie Linux. W ramach pracy zbadałem 4 lasery półprzewodnikowe: dwa krawędziowe oraz dwa VCSEL. Wyznaczyłem dla nich
wartość prądu progowego oraz sprawności, które dobrze zgadzają się z wartościami z katalogu firmy Thorlabs. Otrzymane wyniki
przemawiają, za możliwością wykorzystania mojego programu do badania charakterystyk laserów.

Analizowane lasery krawędziowe charakteryzowały są dużo większą wartością prądu progowego $I_{th}$ niż lasery typu krawędziowego.
Wartość prądu progowego dla laserów krawędziowych rośnie wraz z temperaturą, zależność tą można opisać równaniem~\ref{eq:i_th}.
Zjawisko wzrostu wartości prądu progowego wraz z temperaturą $I_{th}$
spowodowane jest zjawiskami, które można wyjaśnić na podstawie teorii pasmowej ciał stałych.

Drugą badaną cechą laserów półprzewodnikowych była ich sprawność. Wraz ze wzrostem temperatury, sprawność laserów malała.
Taki wyniki zgodny jest z przewidywaniami. Procesy fizyczne odpowiedzialne za to zjawisko można wytłumaczyć m.in. przy pomocy rozkładu Fermiego-Diraca.
\section{Co dalej?}
Możliwa jest dalszy rozwój programu m.in. o sterowanie zasilaniem impulsowym. Dołączenie oscyloskopu do układu pomiarowego pozwoli badać więcej cech
laserów półprzewodnikowych. Komunikacja z oscyloskopem będzie możliwa za pomocą klasy $\mathtt{IODevice.py}$.
