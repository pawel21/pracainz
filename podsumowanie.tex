\chapter{Podsumowanie}
\section{Rezultat pracy}
W ramach pracy stworzyłem program do sterowania pomiarami charakterystyk elektryczno-optycznych laserów półprzewodnikowych.
Program stworzony jest
w dwóch wersjach: skryptowej oraz okienkowej. Program został napisany w języku Python w sposób obiektowy, co ułatwi
pracę nad nim w przyszłości. Praca moja wypełniła lukę, którą był brak programu do sterowania sprzętem firmy Thorlabs
na platformie Linux. W ramach pracy zbadałem 4 lasery półprzewodnikowe: dwa krawędziowe oraz dwa VCSEL. Wyznaczyłem dla nich
wartość prądu progowego oraz sprawności, które dobrze zgadzają się z wartościami z katalogu firmy Thorlabs. Otrzymane wyniki
przemawiają, za możliwością wykorzystania mojego programu do badania charakterystyk laserów. \\

Analizowane lasery krawędziowe charakteryzowały są dużo większą wartością prądu progowego $I_{th}$ niż lasery typu krawędziowego.
Wartość prądu progowego dla laserów krawędziowych rośnie wraz z temperaturą, zależność tą można opisać równaniem~\ref{eq:i_th}.
Natomiast lasery VCSEL charakteryzowały się pewnym prądem minimalnym osiąganym w niższych temperaturach, jest to spotyka ich cecha~\cite{publikacja_1}.
, dla wyższych temperatur wartość prądu progowego również rosła. Zjawisko wzrostu wartości prądu progowego wraz z temperaturą $I_{th}$
spowodowane jest wzrostem współczynnika Auger wraz z temperatura, co obniża wydajność i czas życia ładunku podczas przejścia promienistego.

Drugą badaną cechą laserów półprzewodnikowych była ich sprawność. Wraz ze wzrostem temperatury, sprawność laserów, które testowałem malała.
Taki wyniki zgodny jest z przewidywaniami, proces fizyczny odpowiedzialnym za to zjawisko jest wzrost prawdopodobieństwa obsadzenia
wyższych stanów energetycznych wraz ze wzrostem temperatury w półprzewodniku --- mówi o tym rozkład Fermiego-Dirca.
W wyniku tego energia emitowanych fotonów podczas przejść promienistych maleje.
\section{Co dalej?}
Możliwa jest dalszy rozwój programu m.in. o sterowanie zasilaniem impulsowym. Dołączenie oscyloskopu do układu pomiarowego pozwoli badać więcej cech
laserów półprzewodnikowych. Komunikacja z oscyloskopem będzie możliwa za pomocą klasy $\mathtt{IODevice.py}$.

