\chapter{Podsumowanie}
\section{Rezultat pracy}
Zakładane cele pracy zostały zrealizowane: powstał program do sterowania pomiarami elektryczno-optycznych charakterystyk laserów półprzewodnikowych
oraz zostały zbadane cztery lasery półprzewodnikowe.

Moja praca wypełniła w cześć lukę, którą był brak programu do sterowania sprzętem firmy Thorlabs
na platformie Linux.
Korzystając ze stworzonego programu, dokonałem pomiarów. Następnie na podstawie zebranych danych dokonałem analizy charakterystyk laserów półprzewodnikowych.
W ramach pracy zbadałem cztery lasery półprzewodnikowe: dwa krawędziowe oraz dwa VCSEL. Wyznaczyłem dla nich
wartości prądu progowego oraz sprawności różniczkowe, które dobrze zgadzają się z wartościami z katalogu firmy Thorlabs. Otrzymane wyniki
przemawiają za możliwością wykorzystania mojego programu do badania charakterystyk laserów.

Analizowane lasery krawędziowe charakteryzowały są dużo większą wartością prądu progowego $I_{\mathrm{th}}$ niż lasery typu VCSEL.
Wartość prądu progowego dla laserów krawędziowych rośnie wraz z temperaturą, zależność tą można opisać równaniem~\ref{eq:i_th}.


\section{Co dalej?}
Możliwa jest dalszy rozwój programu m.in. o sterowanie zasilaniem impulsowym. Dołączenie oscyloskopu do układu pomiarowego pozwoli badać więcej cech
laserów półprzewodnikowych. Komunikacja z oscyloskopem będzie możliwa za pomocą klasy $\mathtt{IODevice.py}$.
