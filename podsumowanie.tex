\chapter{Podsumowanie}
\section{Rezultat pracy}
W ramach pracy stworzyłem program do sterowania pomiarami charakterystyk elektryczno-optycznych. Program stworzony jest
w dwóch wersjach: skryptowej oraz okienkowej. Program został napisany w języku Python w sposób obiektowy, co ułatwi
pracę nad nim w przyszłości. Praca moja wypełniła lukę, którą było brak programu do sterowania sprzętem firmy Thorlabs
na platformie Linux. W ramach pracy zbadałem 4 lasery półprzewodnikowe: dwa krawędziowe oraz dwa VCSEL. Wyznaczyłem dla nich
wartość prądu progowego oraz sprawności
\section{Co dalej?}
Możliwa jest dalszy rozwój programu do sterowania m.in. o sterowanie zasilaniem impulsowym.

