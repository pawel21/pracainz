\chapter{Komunikacja z urządzeniami pomiarowymi przy wykorzystaniu oprogramowania open source}
\section{Oprogramowanie open source}
Otwarte oprogramowanie (ang. \textit{open source}) jest odłamem ruch darmowego oprogramowania (ang. \textit{free software}),
którego głównym celem jest możliwość swobodnego dostępu do oprogramowania dla wszystkich. Jawny musi być kod źródłowy.
Przykładami takiego oprogramowania, które wykorzystuje w swojej pracy jest:
\begin{itemize}
\item System Linux (Ubuntu)
\item Język programowania Python
\end{itemize}
\section{Python --- idealne narzędzie dla fizyka}
Język Python jest jednym z najpopularniejszych języków używanych w nauce. Jest on projektem Open Source. Posiada łatwą składnię,
dzięki czemu jest się łatwy w nauce, a programy pisane w nim są przejrzyste. Dzięki ogromnej ilości modułów,
możliwe jest wykonywanie niemalże każdej czynności jakom się potrzebuje. Python używany jest między innymi w
eksperymencie mający za cel znalezienie fal grawitacyjnych. W mojej pracy wykorzystywałem ten język zarówno
do pisania skryptów mających na celu komunikacje i sterowaniem sprzętem laboratoryjnym, jak i wykorzystywałem
do analizy danych uzyskanych w wyniku pomiarów.
Najważniejsze biblioteki, które użyłem do swoich celów to:
\begin{itemize}
\item $\mathtt{Matplotlib}$ ~\cite{matplotlib_book} --- bibliotek do sporządzania wykresów, posiada między innymi możliwość używania symboli \LaTeX.
Możliwość wykonywania animacji co używane do robienia wykresów w czasie rzeczywistym.
\item $\mathtt{Scipy}$ ~\cite{SciPy_book} --- bibliotek do obliczeń numerycznych. Funkcje z niej uzywałem w celu dopasowywania
fukcji dla zebranych danych aby wyznaczyć wartość prądu progowego oraz sprawności.
\item $\mathtt{OS}$ --- bibliotek systemowa. Używana w celu komunikacji ze sprzętem pomiarowym za pomocą wywołań systemowych.
\item $\mathtt{PyQt5}$ --- biblioteka do tworzenia graficznego interfejsu. Użyłem ją aby stworzyć interfejs graficzny za pomocą,
którego możliwe jest sterowanie sprzętem oraz wykonywanie pomiarów charakterystyk laserów półprzewodnikowych
\item $\mathtt{Threading}$ --- biblioteka do tworzenia wątków. Używam jej do robienia wykresów w czasie rzeczywistym.
W tym celu potrzebowałem jeden wątek do komunikuje się ze sprzętem, a drugi w tym samym czasie
na podstawie zebranych danych tworzył wykres w czasie rzeczywistym.
\end{itemize}
Python posiada także biblioteki (np. $\mathtt{Shutil}$) do operowania plikami(jak przenoszenie, usuwanie) co jest często przydatne.
Połączenie bibliotek wymienionych powyżej umożliwiło stworzenia programu, który komunikuje i steruje sprzętem.
Warto jeszcze nadmienić, że aktulalnie rozwijane są dwie wersję Python: Python 2.7 i Python 3. Większą przyszłość ma Python 3,
więc skrypty do analizy danych były pisane w nim. Jednakże biblioteka matplotlib i PyQt5 na chwilę pisania mojej
pracy lepiej współpracowały z Python 2.7, więc interefes pomiarowy został napisany w Python 2.7.
\section{Programowane urządzenia pomiarowe}
Przez programowane urządzenia pomiarowe rozumiemy sprzęt mogący dokonywać pomiarów wielkości elektrycznych i nieelektrycznych,
który wyposzażony jest w interfejs umożliwiający sterowanie nimi przy pomocy komputera. Przykładami takich urządzeń, którymi zajmuje się w swojej pracy są:
\begin{itemize}
\item Zasilacza diód laserowych firmy Thorlabs model LDC4005.
\item Miernik mocy firmy Thorlas firmy Thorlabs model PM100.
\end{itemize}
Z wyżej wymienionymi urządzeniami możliwa jest fizyczna komunikacja za pomocą interfejsu USB przy pomocy standardu komend SCPI,
który zostanie opiszany w dalszej cześci rozdziału.

\section{Komunikacja}
W systemach Unix z którego dziedziczy system Linux, wszystko jest plikiem. Linuksowy sterownik znakowy (ang. \textit{char driver})
pozwala na reprezentowanie urządzenia za pomocą specjalnych plików wirtualnych, które znajdują się w przestrzeni
użytkownika w katalogu $\mathtt{/dev/<nazwa>}$. Obsługa tych plików możliwa jest za pomocą wywołań systemowych (ang. \textit{system call}),
które stanowią API za pomocą którego użytkowniki może sterować sprzętem. Do używania wywołań systemowych potrzebny jest
identyfikator do danego urządzenia, który jest reprezentowany przez deskryptor pliku, będący liczbą całkowitą.
Podstawowe wywołaniami systemowymi pozwalające na sterowanie sprzętem to:
\begin{itemize}
\item $\mathtt{open}$ --- służy do połaczenia z urządzeniem, zwraca deskrypotor pliku.
\item $\mathtt{write}$ --- funkcja służaca do wysyłania komend do urządzenia .
\item $\mathtt{read}$ --- funckja służąca do odczytywania buffora urządzenia.
\item $\mathtt{close}$ --- funkcja zamykająca połączenie.
\end{itemize}
Funkcję te mają swoją implementację w języku C w bibliotece $<\mathtt{fcntl.h}>$ oraz $<\mathtt{unistd.h}>$, oraz w języku Python w bibliotece $\mathtt{os}$.
\section{SCPI --- standard komend do komunikacji z urządzeniami}
SCPI  (ang. \textit{Standard  Commands  for  Programmable  Instruments}) jest tekstowym interfejsem ASCII do programowanych
urządzeń pomiarowych mający na celu standaryzacje polecenie używanych w systemach pomiarowym. Zdefiniowany został 1990 roku,
wedle specyfikacji IEEE 488.2. (Institute of Electrical and Electronics Engineers) --- międzynarodowa organizacja stowarzyszeń
inżynierów elektryków i elektroników.  Dzięki temu możliwa jest obsługa urządzeń pomiarowych przy wykorzystaniu komputera.
Polecenia SCPI są to ciągi tekstowe ASCII, które są wysyłane do urządzenia przez wywołanie systemowe $\mathtt{write}$.
Polecenia są serią jednego lub więcej słów, przy czym wiele z nich używa dodatkowych parametrów.
 Odpowiedzi do zapytania polecenia są zazwyczaj ciągami ASCII.
  W przypadku danych masowych mogą być zwracane także formaty binarne. \\

Cechą poleceń  SCPI jest ich implementacja przez każde urządzenie, czyli to samo polecenie będzie działać na każdym oscyloskopie
 bez względu na producenta. Można wyróżnić dwie grupy poleceń:
\begin{itemize}
\item Polecenia dla każdego urządzenia pomiarowego nie zależnie od jego przeznacznia. Takimi komendami są m.in.
\begin{itemize}
\item $\mathtt{*idn?}$ --- odczytuje identyfikator urządzenia.
\item $\mathtt{*rst}$ --- powoduje przywrócenie ustawień początkowych urządzenia.
\item $\mathtt{*cls}$ --- powdouje wyzerowanie informacji o błędach.
\item $\mathtt{*opc?}$  --- (ang.  operation  complete) jest zapytanie o zakończenie wykonania
poprzedzających poleceń. \\
W  odpowiedzi  na  zapytanie  po  zakończeniu  wykonywania  poprzedzających poleceń urządzenie prześle wartość 1.
\item $\mathtt{*wai}$ ---  (ang.  wait)  oczekiwanie  na  zakończenie  wykonania  poprzedzających poleceń.
\end{itemize}

\item Polecenia charakterystyczne dla danego urządzenia pomiarowego zgodnie z jego przeznaczeniem.
Przykładowe polecenie które będzie działać na każdym zasilaczu korzystającym z standardu SCPI:
\begin{itemize}
\item Służacę do ustawienie wartości prądu na 0.01\,A \\ $\mathtt{SOURce:CURRent:LEVel:AMPLitude}$  $\mathtt{0.01}$
\end{itemize}
\end{itemize}

Fizyczne łącze komunikacyjne nie jest zdefiniowane przez SCPI. Stworzony standard IEEE-488 był dla GPIB,
ale może być również używany z interfejsem RS-232, Ethernet, USB. W przypadku mojej pracy, do komunikacji ze sprzętem
 używam USB.
\newpage